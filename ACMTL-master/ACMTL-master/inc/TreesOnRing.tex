\subsection{基环树}
$O(n+m)$

\texttt{Cir}数组代表环上的点,按照某一顺序给出,\texttt{tot}为环上一共有多少点,用\texttt{preEdge}和\texttt{nextEdge}可以用不同方向遍历环,\texttt{inCir}表示某条边是否是在环上的
\begin{lstlisting}[language=C]
int dfn[MAXN],dfs_clock;
int fa[MAXN],faEdge[MAXN];
int Cir[MAXN],preEdge[MAXN],nextEdge[MAXN],tot;
bool inCir[MAXN];
void GetCir(int x,int father)
{
    dfn[x]=++dfs_clock;
    for(int i=head[x];i;i=next[i])
        if((i^father)!=1)
        {
            if(!dfn[to[i]]) fa[to[i]]=x,faEdge[to[i]]=i,GetCir(to[i],i);
            else if(dfn[to[i]]<dfn[x])
            {
                int p=x;
                Cir[tot=1]=to[i];
                inCir[i]=inCir[i^1]=true;
                preEdge[to[i]]=i^1;
                nextEdge[x]=i;
                while(p!=to[i])
                {
                    int tmp=faEdge[p];
                    Cir[++tot]=p;
                    preEdge[p]=tmp^1;
                    inCir[tmp]=inCir[tmp^1]=true; 
                    p=fa[p];
                    nextEdge[p]=tmp;
                }
            }
        }
}
\end{lstlisting}