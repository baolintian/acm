\subsection{树链剖分}
\begin{lstlisting}[language=C]
int n, m, cnt, sz;
bool vis[100005];
int size[100005], bl[100005], pl[100005], dep[100005];
int last[100005];
int u[100005],v[100005],mark[100005];
int fa[100005], ft[100005];
struct edge{
    int to, next, v;
}e[200005];
int L[400005], R[400005], S[400005], tag[400005];
int find(int x)
{
    return x == fa[x]?x:fa[x] = find(fa[x]);
}
void insert(int u, int v)
{
    e[++cnt].to = v; e[cnt].next = last[u]; last[u] = cnt;
    e[++cnt].to = u; e[cnt].next = last[v]; last[v] = cnt;
}
void dfs1(int x)
{
    vis[x] = size[x] = 1;
    for(int i = last[x]; i; i = e[i].next)
    {
        if(vis[e[i].to])continue;
        dep[e[i].to] = dep[x] + 1;
        ft[e[i].to] = x;       
        dfs1(e[i].to);
        size[x] += size[e[i].to];
    }
}
void dfs2(int x, int chain)
{
    pl[x] = ++sz; bl[x] = chain;
    int k = 0;
    for(int i = last[x]; i; i = e[i].next)
        if(dep[e[i].to] > dep[x] && size[k] < size[e[i].to])
            k = e[i].to;
    if(!k)return;
    dfs2(k, chain);
    for(int i = last[x]; i; i = e[i].next)
        if(dep[e[i].to] > dep[x] && k != e[i].to)
            dfs2(e[i].to, e[i].to);
}
void build(int k, int l, int r)
{
    L[k] = l; R[k] = r; tag[k] = 0;
    if(l == r)
    {
        if(l != 1)S[k] = 1;
        else S[k] = 0;
        return;
    }
    int mid = (l + r) >> 1;
    build(k << 1, l, mid);
    build(k << 1 | 1, mid + 1, r);
    S[k] = S[k << 1] + S[k << 1 | 1];
}
void change(int k, int x, int y)
{
    if(!S[k])return;
    int l = L[k], r = R[k], mid = (l + r) >> 1;
    if(l == x && y == r)
    {
        S[k] = 0, tag[k] = 1; return;
    }
    if(mid >= y)change(k << 1, x, y);
    else if(mid < x)change(k << 1 | 1, x, y);
    else
    {
        change(k << 1, x, mid);
        change(k << 1 | 1, mid + 1, y);
    }
    S[k] = S[k << 1] + S[k << 1 | 1];
}
int query(int k, int x, int y)
{
    if(!S[k])return 0;
    int l = L[k], r = R[k], mid = (l + r) >> 1;
    if(l == x && y == r)
        return S[k];
    if(mid >= y)return query(k << 1, x, y);
    else if(mid < x)return query(k << 1 | 1, x, y);
    else
        return query(k << 1, x, mid) + query(k << 1 | 1, mid + 1, y);
}
int solvesum(int x, int y)
{
    if(x == y)return 0;
    int sum = 0;
    while(bl[x] != bl[y])
    {
        if(dep[bl[x]]<dep[bl[y]])swap(x,y);
        if(!vis[x])sum += query(1, pl[bl[x]], pl[x]);
        x = ft[bl[x]];
    }
    if(pl[x] > pl[y])swap(x, y);    
    if(pl[x] + 1 <= pl[y])
        sum += query(1, pl[x] + 1, pl[y]);
    return sum;
}
void add(int x, int y)
{
    if(x == y)return;
    while(bl[x] != bl[y])
    {
        if(dep[bl[x]]<dep[bl[y]])swap(x,y);
        if(!vis[x])
        {
            change(1, pl[bl[x]], pl[x]);
            fa[find(x)] = find(pl[x]);
            vis[x] = 1;
        }
        x = ft[bl[x]];
    }
    if(pl[x] > pl[y])swap(x, y);    
    if(pl[x] + 1 <= pl[y])
        change(1, pl[x] + 1, pl[y]);
}
int main()
{
    int Tes = F();
    for(int cas = 1; cas <= Tes; cas++)
    {
        printf("Case #%d:\n", cas);
        cnt = sz = 0;
        n = F(); m = F();
        for(int i = 1; i <= n; i++)
            last[i] = vis[i] = 0;
        for(int i = 1; i <= m; i++)
            mark[i] = 0;
        for(int i = 1; i <= n; i++)fa[i] = i;
        for(int i = 1; i <= m; i++)
        {
            u[i] = F(); v[i] = F();
            if(find(u[i]) != find(v[i]))
            {
                mark[i] = 1;
                insert(u[i], v[i]);
            }
            fa[find(u[i])] = find(v[i]);
        }
        build(1,1,n);
        dfs1(1);
        dfs2(1,1);
        for(int i = 1; i <= n; i++)vis[i] = 0;
        for(int i = 1; i <= m; i++)
            if(!mark[i])
                add(u[i], v[i]);
        int Q;
        Q = F();
        while(Q--)
        {
            int opt, u, v;
            opt = F(); u = F(); v = F();
            if(opt == 1)add(u, v);
            else
            {
                int ans = solvesum(u, v);
                printf("%d\n", ans);
            }
        }
    }
    return 0;
}
\end{lstlisting} 
