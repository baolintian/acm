\subsection{半平面交}
\begin{lstlisting}[language=C]
#include<iostream>
#include<cstdio>
#include<algorithm>
#include<string>
#include<cstring>
#include<queue>
#include<vector>
#include<cmath>
using namespace std;
const int MAXN=100010;
const double eps=1e-16;
int T,n,cnt,front,rear;
double area;
int dcmp(double x)
{
	if(fabs(x)<eps) return 0;
	return x>eps?1:-1;
}
struct Point
{
	double x,y;
	Point(){}
	Point(double x,double y):x(x),y(y){}
}p[MAXN];
typedef Point Vector;
Vector operator+(const Vector &p,const Vector &q){return Vector(p.x+q.x,p.y+q.y);}
Vector operator-(const Vector &p,const Vector &q){return Vector(p.x-q.x,p.y-q.y);}
Vector operator*(const Vector &p,double k){return Vector(p.x*k,p.y*k);}
double Cross(const Vector &p,const Vector &q){return p.x*q.y-p.y*q.x;}
struct Line
{
	Point p;
	Vector v;
	double ang;
	Line(){}
	Line(const Point &p,const Vector &v):p(p),v(v){ang=atan2(v.y,v.x);}
	friend bool operator<(const Line &p,const Line &q){return p.ang<q.ang;}
}L[MAXN],q[MAXN];
Point LineIntersection(const Line &a,const Line &b)
{
	Vector p=a.p-b.p;
	double t=Cross(b.v,p)/Cross(a.v,b.v);
	return Point(a.p+a.v*t);
}
bool Left(const Point &a,const Line &b){return dcmp(Cross(a-b.p,b.v))<0;}
bool Halfplane()
{
	sort(L+1,L+n+1);
	q[front=rear=1]=L[1];
	for(int i=2;i<=n;i++)
	{
		while(front<rear&&!Left(p[rear-1],L[i])) rear--;
		while(front<rear&&!Left(p[front],L[i])) front++;
		q[++rear]=L[i];
		if(!dcmp(q[rear].ang-q[rear-1].ang)&&Left(L[i].p,q[--rear])) q[rear]=L[i];
		if(front<rear) p[rear-1]=LineIntersection(q[rear-1],q[rear]);
	}
	while(front<rear&&!Left(p[rear-1],q[front])) rear--;
	if(rear-front<=1) return false;
	p[rear]=LineIntersection(q[rear],q[front]);
	return true;
}
int main()
{
	scanf("%d",&T);
	for(int cas=1;cas<=T;cas++)
	{
		scanf("%d",&n);
		for(int i=1;i<=n;i++) scanf("%lf%lf",&p[i].x,&p[i].y);
		for(int i=1;i<n;i++) L[++cnt]=Line(p[i],p[i+1]-p[i]);
		L[++cnt]=Line(p[n],p[1]-p[n]);
	}
	n=cnt;
	if(!Halfplane()) {puts("0.000");return 0;}
	else
	{
		for(int i=front;i<rear;i++) area+=Cross(p[i],p[i+1]);
		area+=Cross(p[rear],p[front]);
	}
	printf("%.03lf\n",fabs(area/2));
	return 0;
}
\end{lstlisting}