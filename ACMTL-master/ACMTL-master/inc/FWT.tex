\subsection{FWT}
求N个小于M的素数异或起来等于0的方案数

Xor:

tf(f) = (tf(f0 - f1), tf (f0 + f1))

utf(h) = (utf((f0 + f1)/2), utf((f1 - f0)/2))

And:

tf(f) = (tf(f0 + f1), tf(f1))

utf(h) = (utf((f0 - f1)/2), utf(f1))

Or:

tf(f) = (tf(f0), tf(f0 + f1))

utf(h) = (utf(f0), utf(f1 - f0))
\begin{lstlisting}[language=C]
int T;
int n, m, L;
ll ine;
bool mark[50005];
ll A[65536];
ll qpow(ll a, ll b)
{
    ll ans = 1;
    for(ll i = b; i; i >>= 1, a = a * a % mod)
        if(i & 1)ans = ans * a % mod;
    return ans;
}
void fwt(ll *A, int n)
{
    if(n == 1)return;
    int h = n >> 1;
    fwt(A, h); fwt(A + h, h);
    for(int i = 0; i < h; i++)
    {
        ll u = A[i], v = A[i + h];
        A[i] = (u + v) % mod;
        A[i + h] = (u - v + mod) % mod;
    }
}
void ifwt(ll *A, int n)
{
    if(n == 1)return;
    int h = n >> 1;
    for(int i = 0; i < h; i++)
    {
        ll u = A[i], v = A[i + h];
        A[i] = (u + v) * ine % mod;
        A[i + h] = ((u - v + mod) % mod) * ine % mod;
    }
    ifwt(A, h); ifwt(A + h, h);
}
int main()
{
    mark[1] = 1;
    for(int i = 2; i <= 50000; i++)
        if(!mark[i])
            for(int j = i + i; j <= 50000; j += i)
                mark[j] = 1;
    T = read();
    ine = qpow(2, mod - 2);
    while(T--)
    {
        memset(A, 0, sizeof(A));
        n = read(); m = read();
        for(L = 1; L <= m; L <<= 1);
        for(int i = 1; i <= m; i++)
            if(!mark[i])A[i] = 1;
        fwt(A, L);
        for(int i = 0; i < L; i++)A[i] = qpow(A[i], n);
        ifwt(A, L);
        cout << A[0] << endl;
    }
    return 0;
}
\end{lstlisting}