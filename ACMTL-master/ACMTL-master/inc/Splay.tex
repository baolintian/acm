\subsection{Splay}
\begin{lstlisting}[language=C] 
//序列翻转
int n,ans,rt;
int a[300005];
int c[300005][2],fa[300005],size[300005],val[300005];
bool rev[300005];
void update(int k)
{
	int l=c[k][0],r=c[k][1];
	size[k]=size[l]+size[r]+1;
}
void rotate(int x,int &k)
{
	int y=fa[x],z=fa[y],l,r;
	if(c[y][0]==x)l=0;else l=1;r=l^1;
	if(y==k)k=x;
	else {if(c[z][0]==y)c[z][0]=x;else c[z][1]=x;}
	fa[x]=z;fa[y]=x;fa[c[x][r]]=y;
	c[y][l]=c[x][r];c[x][r]=y;
	update(y);update(x);
}
void splay(int x,int &k)
{
	while(x!=k)
	{
		int y=fa[x],z=fa[y];
		if(y!=k)
		{
			if(c[y][0]==x^c[z][0]==y)rotate(x,k);
			else rotate(y,k);
		}
		rotate(x,k);
	}
}
void pushdown(int k)
{
	int l=c[k][0],r=c[k][1];
	rev[k]^=1;rev[l]^=1;rev[r]^=1;
	swap(c[k][0],c[k][1]);
}
void build(int l,int r,int f)
{
	if(l>r)return;
	int mid=(l+r)>>1;
	if(mid<f)c[f][0]=mid;
	else c[f][1]=mid;
	size[mid]=1;val[mid]=a[mid];fa[mid]=f;
	if(l==r)return;
	build(l,mid-1,mid);build(mid+1,r,mid);
	update(mid);
}
int find(int k,int rk)
{
	if(rev[k])pushdown(k);
	int l=c[k][0],r=c[k][1];
	if(size[l]+1==rk)return k;
	else if(size[l]>=rk)return find(l,rk);
	else return find(r,rk-size[l]-1);
}
void rever(int l,int r)
{
	int x=find(rt,l),y=find(rt,r+2);
	splay(x,rt);splay(y,c[x][1]);
	int z=c[y][0];
	rev[z]^=1;
}
int main()
{
	n=read();
	for(int i=1;i<=n;i++)
		a[i+1]=read();
	build(1,n+2,0);rt=(n+3)>>1;
	while(val[find(rt,2)]!=1)
	{
		ans++;
		rever(1,val[find(rt,2)]);
		if(ans>100000)
		{
			puts("-1");return 0;
		}
	}
	printf("%d\n",ans);
	return 0;
}
\end{lstlisting}