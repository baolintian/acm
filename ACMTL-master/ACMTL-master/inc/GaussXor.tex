\subsection{异或方程组高斯消元}
用高斯消元法解异或方程组,求秩
\begin{lstlisting}[language=C]
bool Gauss_Jordan(bool A[SIZE][SIZE],int n,int m){//是否有解
	//行数0~n-1,列数0~m(m这一列是常数)
	for(int i=0;i<n&&i<m;i++){
		int p=i;
		for(int k=i+1;k<n;k++) if(A[k][i]>A[p][i]) p=k;
		if(!A[p][i]) continue;
		for(int k=0;k<=m;k++) swap(A[i][k],A[p][k]);
		for(int j=0;j<n;j++){
			if(j==i||!A[j][i]) continue;
			for(int k=0;k<=m;k++) A[j][k]^=A[i][k];
		}
	}
	//两种0=1的姿势
	for(int i=0;i<n&&i<m;i++)
		if(A[i][i]==0&&A[i][m]!=0) return false;
	for(int i=m;i<n;i++)
		if(A[i][m]!=0) return false;
	return true;
}
int xor_rank(ll A[SIZEN][SIZEN],int b[SIZEN],int n,int m){
	//使用高斯约当消元法,求异或方程组的秩
	//A是异或方程组,b是方程右边的常数向量,n是行数,m是列数
	int i,j,k;
	for(i=1;i<=n;i++){//第i次消元
		for(k=i;k<=m;k++){//选取消元列
			for(j=i;j<=n;j++){//选取消元行
				if(A[j][k]) goto BREAK;
			}
		}
		BREAK:;
		if(j==n+1) break;//找不到可消元的元素了
		if(i!=j){//交换行
			for(int km=1;km<=m;km++) swap(A[i][km],A[j][km]);
			swap(b[i],b[j]);
		}
		if(i!=k) for(int km=1;km<=n;km++) swap(A[km][i],A[km][k]);//交换列
		for(j=1;j<=n;j++){//消元
			if(j==i) continue;
			if(A[j][i]){
				for(int km=1;km<=m;km++) A[j][km]^=A[i][km];
				b[j]^=b[i];
			}
		}
	}
	return i-1;
}
\end{lstlisting}
