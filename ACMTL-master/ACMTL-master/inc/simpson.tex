\subsection{辛普森积分}
\begin{lstlisting}[language=C] 
#include<map>
#include<cmath>
#include<ctime>
#include<queue>
#include<cstdio>
#include<vector>
#include<bitset>
#include<cstring>
#include<iostream>
#include<algorithm>
#define eps 1e-8
#define pa pair<int,int>
#define ll long long
#define inf 1000000000
using namespace std;
struct P{
    double x, y;
    P(){}
    P(double _x, double _y):x(_x), y(_y){}
    friend P operator-(P a, P b){
        return P(a.x - b.x, a.y - b.y);
    }
    friend double operator*(P a, P b){
        return a.x * b.y - a.y * b.x;
    }
}p[105];
struct L{
    P a, b;
}l[105];
vector<double> ve;
int n;
double R;
void add(double X, L l)
{
    if(l.a.x < X - eps && l.b.x < X - eps)return;
    if(l.a.x > X + eps && l.b.x > X + eps)return;
    if(fabs(X - l.a.x) < eps && fabs(X - l.b.x) < eps)return;
    double t;
    if(fabs(l.b.x - X) < eps)t = l.b.y;
    else t = (X - l.a.x) / (l.b.x - X);    
    ve.push_back(l.a.y + (l.b.y - l.a.y) * t / (t + 1));
}
double F(double X)
{
    if(abs(X) > R)return 0;
    ve.clear();
    for(int i = 1; i <= n; i++)
        add(X, l[i]);
    double U = sqrt(R * R - X * X), ANS = 0;
    sort(ve.begin(), ve.end());
    bool flag = 0;
    for(int i = 0; i < (int)ve.size() - 1; i++)
    {
        flag ^= 1;
        if(ve[i] > U)break;
        if(flag)
            ANS += min(ve[i + 1], U) - ve[i];
    }
    return ANS;
}
double cal(double fl, double fm, double fr, double L)
{
    return (fl + fm * 4 + fr) * L / 6;
}
double simpson(double l, double r, double fl, double fm, double fr, double S, int dep)
{
    double mid = (l + r) / 2;
    double m1 = (l + mid) / 2, m2 = (r + mid) / 2;
    double f1 = F(m1), f2 = F(m2);
    double g1 = cal(fl, f1, fm, mid - l), g2 = cal(fm, f2, fr, r - mid);
    if(fabs(g1 + g2 - S) < 1e-12 && dep >= 10)return g1 + g2;
    return simpson(l, mid, fl, f1, fm, g1, dep + 1) + simpson(mid, r, fm, f2, fr, g2, dep + 1);    
}
double solve(double l, double r)
{
    if(l > r)return 0;
    double mid = (l + r) / 2;
    return simpson(l, r, 0, F(mid), 0, (F(l) + F(r) + F(mid) * 4) * (r - l) / 6, 0);
}
int main()
{
    scanf("%d%lf", &n, &R);
    for(int i = 1; i <= n; i++)
        scanf("%lf%lf", &p[i].x, &p[i].y);
    p[n + 1] = p[1];
    for(int i = 1; i <= n; i++)
        l[i].a = p[i], l[i].b = p[i + 1];
    double ANS = solve(-R, -2) + solve(-2, 1) + solve(1, R);
    printf("%.10lf\n", ANS);
    return 0;
}
\end{lstlisting} 
