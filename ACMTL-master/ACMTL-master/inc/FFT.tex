\subsection{FFT}
\begin{lstlisting}[language=C]
const int SIZEN=800010,BASE=10;
const double PI=acos(-1.0),eps=1e-6;
class Complex{//复数
public:
	double x,y;//x+yi
	Complex(double x_=0,double y_=0){x=x_,y=y_;}
	void print(void){cout<<"("<<x<<","<<y<<")";}
	void clear(void){x=y=0;}
};
void print(Complex a){printf("%.8lf+%.8lfi",a.x,a.y);}
void swap(Complex &a,Complex &b){Complex c=a;a=b,b=c;}
Complex operator + (Complex a,Complex b){return Complex(a.x+b.x,a.y+b.y);}
Complex operator - (Complex a,Complex b){return Complex(a.x-b.x,a.y-b.y);}
Complex operator * (Complex a,Complex b){return Complex(a.x*b.x-a.y*b.y,a.x*b.y+b.x*a.y);}
Complex operator * (Complex a,double b){return Complex(a.x*b,a.y*b);}
Complex operator / (Complex a,double b){return Complex(a.x/b,a.y/b);}
class Poly{
public:
	int n;
	Complex s[SIZEN];
	void Initialize(char str[]){
		n=strlen(str);
		for(int i=0;i<n;i++) s[i]=Complex(str[n-1-i]-'0',0);
	}
	void Read(void){
		static char str[SIZEN];
		scanf("%s",str);
		Initialize(str);
	}
	void Assign(char str[]){
		static int a[SIZEN];
		int len;
		for(int i=0;i<n;i++) a[i]=int(s[i].x+0.5);
		for(len=0;len<n||a[len];len++){
			a[len+1]+=a[len]/BASE;
			a[len]%=BASE;
		}
		while(len>1&&!a[len-1]) len--;
		for(int i=0;i<len;i++) str[i]=a[len-1-i]+'0';
		str[len]=0;
	}
	void Print(void){
		static char str[SIZEN];
		Assign(str);
		printf("%s\n",str);
	}
	void Rader_Transform(void){
		int j=0,k;
		for(int i=0;i<n;i++){
			if(j>i) swap(s[i],s[j]);
			k=n;
			while(j&(k>>=1)) j&=~k;
			j|=k;
		}
	}
	void FFT(bool type){//type=1是求值(系数求点),type=0是插值(点求系数)
		Rader_Transform();//不能忘了
		double pi=type?PI:-PI;
		Complex w0;
		for(int step=1;step<n;step<<=1){
			double alpha=pi/step;
			Complex wn(cos(pi/step),sin(pi/step)),wk(1.0,0.0);
			for(int k=0;k<step;k++){
				for(int Ek=k;Ek<n;Ek+=step<<1){
					int Ok=Ek+step;
					Complex t=wk*s[Ok];
					s[Ok]=s[Ek]-t;
					s[Ek]=s[Ek]+t;
				}
				wk=wk*wn;
			}
		}
		if(!type) for(int i=0;i<n;i++) s[i]=s[i]/n;
	}
	void operator *= (Poly &b){
		int S=1;while(S<n+b.n) S<<=1;
		n=b.n=S;
		FFT(true);
		b.FFT(true);
		for(int i=0;i<n;i++) s[i]=s[i]*b.s[i];
		FFT(false);
	}
};
\end{lstlisting}
\subsection{FFT for zyh}
$O(k \log k)$

首先将$k$赋值为大于两个多项式中最大的次数界的最小的$2$的幂次的二倍。$v = 0$为DFT,$v = 1$为IDFT,不需要在结束后除以$k$,该段已在$v=1$时处理。

先调用一次\texttt{initFFT(k)}。
\begin{lstlisting}[language=C]
const double PI=2*asin(1);
struct P{double x,y;};
P operator+(const P &i,const P &j){return (P){i.x+j.x,i.y+j.y};}
P operator-(const P &i,const P &j){return (P){i.x-j.x,i.y-j.y};}
P operator*(const P &i,const P &j){double p=i.x*j.x,q=i.y*j.y,r=(i.x+i.y)*(j.x+j.y);return (P){p-q,r-p-q};}
P operator/(const P &i,int j){return (P){i.x/j,i.y/j};}
int k;
P w[2][MAXK];
void initFFT(int k)
{
    for(int i=0;i<=k;i++)
        w[0][i]=w[1][k-i]=(P){cos(2*i*PI/k),sin(2*i*PI/k)};
}
void FFT(P x[],int k,int v)
{
	int i,j,l;P tmp;
	for(i=j=0;i<k;i++)
	{
		if(i>j) swap(x[i],x[j]);
		for(l=k>>1;(j^=l)<l;l>>=1);
	}
	for(i=2;i<=k;i<<=1)
		for(j=0;j<k;j+=i)
			for(l=0;l<(i>>1);l++)
			{
				tmp=x[j+l+(i>>1)]*w[v][k/i*l];
				x[j+l+(i>>1)]=x[j+l]-tmp;
				x[j+l]=x[j+l]+tmp;
			}
    if(v)
    {
        for(int i=0;i<k;i++)
            x[i]=x[i]/k;
    }
}
\end{lstlisting}