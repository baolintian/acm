\subsection{后缀数组}
后缀排序,输出SA数组以及height数组
\begin{lstlisting}[language=C] 
int n,p,q=1,k;
char ch[100005];
int sa[2][100005],rk[2][100005];
int a[100005],h[100005],v[100005];
void calsa(int *sa,int *rk,int *SA,int *RK)
{
	for(int i=1;i<=n;i++)v[rk[sa[i]]]=i;
	for(int i=n;i;i--)
		if(sa[i]>k)
			SA[v[rk[sa[i]-k]]--]=sa[i]-k;
	for(int i=n-k+1;i<=n;i++)
		SA[v[rk[i]]--]=i;
	for(int i=1;i<=n;i++)
		RK[SA[i]]=RK[SA[i-1]]+(rk[SA[i-1]]!=rk[SA[i]]||rk[SA[i-1]+k]!=rk[SA[i]+k]);
}
void getsa()
{
	for(int i=1;i<=n;i++)v[a[i]]++;
	for(int i=1;i<=30;i++)v[i]+=v[i-1];
	for(int i=1;i<=n;i++)
		sa[p][v[a[i]]--]=i;
	for(int i=1;i<=n;i++)
		rk[p][sa[p][i]]=rk[p][sa[p][i-1]]+(a[sa[p][i]]!=a[sa[p][i-1]]);
	for(k=1;k<n;k<<=1,swap(p,q))
		calsa(sa[p],rk[p],sa[q],rk[q]);
	for(int i=1,k=0;i<=n;i++)
	{
		int j=sa[p][rk[p][i]-1];
		while(a[i+k]==a[j+k])k++;
		h[rk[p][i]]=k;if(k)k--;
	}
}
\end{lstlisting}