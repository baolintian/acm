\subsection{link cut tree}
矩阵树定理
1、 G 的度数矩阵 D[G]是一个 n*n 的矩阵, 并且满足: 当 i≠j 时,dij=0;当 i=j时, dij 等于 vi 的度数。

2、 G 的邻接矩阵 A[G]也是一个 n*n 的矩阵, 并且满足: 如果 vi、 vj 之间有边直接相连,则 aij=1,否则为 0。 

我们定义 G 的 Kirchhoff 矩阵(也称为拉普拉斯算子)C[G]为 C[G]=D[G]-A[G],则 Matrix-Tree 定理可以描述为:

G 的所有不同的生成树的个数等于其 Kirchhoff矩阵 C[G]任何一个 n-1 阶主子式的行列式的绝对值。
\begin{lstlisting}[language=C]
ll a[105][105];
int n, m, id[15][15];
char ch[15][15];
int xx[4] = {1, -1, 0, 0}, yy[4] = {0, 0, 1, -1};
int gauss(int n)
{
    if(n < 0)return 0;
    ll det = 1;
    int flag = 1;
    for(int i = 1; i <= n; i++)
        for(int j = 1; j <= n; j++)
            a[i][j] = (a[i][j] % mod + mod) % mod;
    for(int i = 1; i <= n; i++)
    {
        for(int j = i + 1; j <= n; j++)
            while(a[j][i])
            {
                ll tmp = a[i][i] / a[j][i];
                for(int k = i; k <= n; k++)
                    a[i][k] = ((a[i][k] - a[j][k] * tmp) % mod + mod) % mod;
                swap(a[i], a[j]);
                flag = -flag;
            }        
        if(!a[i][i])return 0;
        det = det * a[i][i] % mod;
    }
    if(flag == -1)return mod - det;
    return det;
}
int main()
{
    while(1)
    {
        memset(a, 0, sizeof(a));
        n = read(); m = read();
        if(n == 0)break;
        for(int i = 1; i <= n; i++)
            scanf("%s", ch[i] + 1);
        int tmp = 0;
        for(int i = 1; i <= n; i++)
            for(int j = 1; j <= m; j++)
                if(ch[i][j] != '*')
                    id[i][j] = ++tmp;
        for(int i = 1; i <= n; i++)
            for(int j = 1; j <= m; j++)
                if(ch[i][j] != '*')
                    for(int k = 0; k < 4; k++)
                    {
                        int x = i + xx[k], y = j + yy[k];
                        int u = id[i][j], v = id[x][y];
                        if(x > 0 && y > 0 && x <= n && y <= m && ch[x][y] != '*')
                            a[u][v]--, a[u][u]++;
                    }
        cout << gauss(tmp - 1) << endl;
    }
    return 0;
}
\end{lstlisting} 
