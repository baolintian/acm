\subsection{曼哈顿最小生成树}
\begin{lstlisting}[language=C]
#include <iostream>  
#include <cstdio>  
#include <algorithm>  
#define lowbit(x) (x&(-x))  
using namespace std;  
const int N = 100005;  
struct Point{  
    int x,y,id;  
    bool operator<(const Point p)const{  
        return x!=p.x?x<p.x:y<p.y;  
    }  
}p[N];  
struct BIT{  
    int min_val,pos;  
    void init(){  
        min_val=(1<<30);  
        pos=-1;  
    }  
}bit[N];  
struct Edge{  
    int u,v,d;  
    bool operator<(const Edge e)const{  
        return d<e.d;  
    }  
}e[N<<2];  
int n,tot,pre[N];  
int find(int x){  
    return pre[x]=(x==pre[x]?x:find(pre[x]));  
}  
int dist(int i,int j){  
    return abs(p[i].x-p[j].x)+abs(p[i].y-p[j].y);  
}  
void addedge(int u,int v,int d){  
    e[tot].u=u;  
    e[tot].v=v;  
    e[tot++].d=d;  
}  
void update(int x,int val,int pos){  
    for(int i=x;i>=1;i-=lowbit(i))  
        if(val<bit[i].min_val)  
            bit[i].min_val=val,bit[i].pos=pos;  
}  
int ask(int x,int m){  
    int min_val=(1<<30),pos=-1;  
    for(int i=x;i<=m;i+=lowbit(i))  
        if(bit[i].min_val<min_val)  
            min_val=bit[i].min_val,pos=bit[i].pos;  
    return pos;  
}  
int k;  
int Manhattan_minimum_spanning_tree(int n,Point *p){  
    int a[N],b[N];  
    for(int dir=0;dir<4;dir++){  
        //4种坐标变换  
        if(dir==1||dir==3){  
            for(int i=0;i<n;i++)  
                swap(p[i].x,p[i].y);  
        }  
        else if(dir==2){  
            for(int i=0;i<n;i++){  
                p[i].x=-p[i].x;  
            }  
        }  
        sort(p,p+n);  
        for(int i=0;i<n;i++){  
            a[i]=b[i]=p[i].y-p[i].x;  
        }  
        sort(b,b+n);  
        int m=unique(b,b+n)-b;  
        for(int i=1;i<=m;i++)  
            bit[i].init();  
        for(int i=n-1;i>=0;i--){  
            int pos=lower_bound(b,b+m,a[i])-b+1;   //BIT中从1开始  
            int ans=ask(pos,m);  
            if(ans!=-1)  
                addedge(p[i].id,p[ans].id,dist(i,ans));  
            update(pos,p[i].x+p[i].y,i);  
        }  
    }  
    sort(e,e+tot);  
    int cnt=n-k;  
    for(int i=0;i<n;i++)  
        pre[i]=i;  
    for(int i=0;i<tot;i++){  
        int u=e[i].u,v=e[i].v;  
        int fa=find(u),fb=find(v);  
        if(fa!=fb){  
            cnt--;  
            pre[fa]=fb;  
            if(cnt==0)  
                return e[i].d;  
        }  
    }  
}
\end{lstlisting}